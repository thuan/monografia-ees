\chapter{Introdução} % (fold)
Devido a constante evolução das redes de comunicação e ao surgimento continuo de
novas tecnologias, as tecnologias móveis tornaram-se cada vez mais importantes e
presentes no cotidiano das pessoas, sendo utilizadas em pesquisas avançadas no
meio acadêmico, corporativo, no entretenimento e no auxilio as atividades
pessoais dos seus usuários. Novas vertentes relacionadas a usabilidade
destas tecnologias móveis surgiram principalmente pela evolução das redes móveis,
algo que está sendo viabilizado principalmente pela crescente popularização e
diversificação dos dispositivos móveis.

Um dispositivo móvel, designado popularmente em inglês por \textit{handheld}, é
considerado um computador de bolso por possuir interfaces de entrada e saída
de dados, \textit{(input/output)}, uma tela \textit{(display)} e um teclado
especificamente. Entretanto os dispositivos móveis como \textit{palm-tops}, \textit{tablets}
e \textit{smartphones} provem uma solução integrada entre estas duas interfaces,
a interface \textit{touch screen}.

Os dispositivos móveis hoje possuem soluções para múltiplas finalidades, proporcionando
ao usuário final comunicação através de serviços de voz e dados, desde serviços
mais simples como envio de mensagens SMS\textit{(Short Message Service)}
até a serviços mais complexos, como as webconferências, que e exigem conectividade
com a \textit{internet} e demandam um intenso fluxo de dados para
disponibilização de serviços conhecidos como \textit{media streaming}.

Considerando estas potencialidades de uso e à diversificação de soluções em uma
escala de disponibilidade crescente, os \textit{smartphones} e \textit{tablets}
assumiram papel central nos últimos anos. Dados da Anatel indicam que o Brasil
terminou o ano de 2012 com 242,2 milhões de dispositivos móveis conectáveis
a uma densidade de 123,9 disp/100 hab. Estes números são muito expressivos,
principalmente considerando a crescente demanda de acesso à informação pela internet móvel.

Estes aparelhos evoluíram e se diversificaram muito nos últimos anos, o que possibilitou a criação de
excelentes sistemas operacionais para eles. Alguns fabricantes, como a Apple e a Google, lançaram
ferramentas de desenvolvimento de software, para permitir que outras empresas e desenvolvedores
independentes possam desenvolver aplicativos para as respectivas plataformas.
Nesse sentido, a motivação desta pesquisa se baseia numa investigação das limitações existentes
acerca do desenvolvimento de aplicações mobile para as plataformas móveis. Pois, no contexto de alguns
sistemas operacionais móveis, este desenvolvimento pode se tornar uma atividade complexa desde a fase
inicial devido ao uso de ferramentas específicas e API’s (Interface de Programação de Aplicações) para
escrever o código em diferentes plataformas. Muitas vezes se torna complexo para os programadores
entenderem o que é preciso para desenvolver e distribuir uma determinada aplicação que implemente
serviços Web para um dispositivo específico. Como cada plataforma tem diferentes processos e requisitos
para a adesão destas aplicações, a documentação para partes diferentes do processo de desenvolvimento são
muitas vezes dispersas e difíceis de agrupar e sintetizar. Isto substancialmente caracteriza problemas
consideráveis quanto à interoperabilidade e convergência entre essas plataformas de desenvolvimento. Para
uma melhor compreensão deste estudo, este trabalho está dividido em duas contextualizações importantes: a
caracterização do chamado ecossistema móvel e as principais diferenças e limitações que a diversidade de
tecnologias impõe acerca do desenvolvimento de aplicações para dispositivos móveis.

Pretende-se, ao longo da pesquisa, descrever as características principais de um ecossistema móvel, as
relações existentes entre os sistemas operacionais móveis e o desenvolvimento de softwares para oss
mesmos, fazendo considerações tanto a respeito das aplicações nativas como das aplicações
multiplataforma, que proporcionam maior interoperabilidade ao usuário final. Portanto, serão abordadas as
principais características de algumas tecnologias (frameworks) importantes, que atualmente estão sendo
utilizadas nessa área de pesquisa, bem como serão também expostas algumas necessidades específicas que
podem contribuir para uma melhor convergência de soluções em softwares mobile, especialmente no que diz
respeito ao desenvolvimento multiplataforma, traçando parâmetros e requisitos segundo as recomendações
da W3C (World Wide Web Consortium).




(Motivação) Nos dias atuais, estamos vivenciando a chamada era da mobilidade, onde dispositivos
móveis estão cada vez mais presentes no cotidiano das pessoas, com isso a
diversificação existente no chamado ecossistema móvel, se destaca para nós,
usuários finais, nos sistemas operacionais existentes sejam eles Android, iOS,
Windows Phone e o novo Tizen. \cite{adriel2012titanium}

(Problemática) Durante o processo de desenvolvimento de uma aplicação mobile o
fator inicial a ser levado em consideração é o sistema operacional a ser adotado.
Cada sistema operacional possui uma plataforma de desenvolvimento, ou seja, as
linguagens e os frameworks de desenvolvimento são distintos em cada plataforma,
essa diversificação torna-se um problema quando pensamos em portabilidade,
ou seja, desenvolver uma única aplicação que seja portável (usável) em
diferentes plataformas.
Muitas linguagens Android java, objective C, web, Mojo, .net

(Objetivos) Para solucionar esse problema, nossos estudos foram realizados no
chamado, desenvolvimento mobile multiplataforma, que utilizam frameworks que
proporcionam uma experiência transparente ao usuário final utilizando-se de
diferentes plataformas de desenvolvimento...

(Visão Geral dos capítulos)...
Lorem ipsum dolor sit amet, consectetur adipisicing elit, sed do eiusmod
tempor incididunt ut labore et dolore magna aliqua. Ut enim ad minim veniam,
quis nostrud exercitation ullamco laboris nisi ut aliquip ex ea commodo
consequat. Duis aute irure dolor in reprehenderit in voluptate velit esse
cillum dolore eu fugiat nulla pariatur. Excepteur sint occaecat cupidatat non
proident, sunt in culpa qui officia deserunt mollit anim id est laborum.


% chapter introducao (end)