\chapter{Introdução} % (fold)
\label{cha:chapter_name}

\section{Contextualização} % (fold)
\label{sec:contextualizacao}
Devido a constante evolução das redes de comunicação e ao surgimento continuo de
novas tecnologias, as chamadas tecnologias móveis, tornaram-se cada vez mais
presentes no cotidiano das pessoas, sendo utilizadas em pesquisas avançadas nos
meios acadêmicos, corporativos, no entretenimento e no auxílio as atividades
pessoais de cada usuário. Essas vertentes relacionadas a usabilidade
das tecnologias móveis surgiram principalmente pela evolução das redes
móveis, algo que está sendo viabilizado principalmente pela crescente difusão,
popularização e diversificação destas tecnologias.

Primordialmente um dispositivo móvel, designado popularmente em inglês pelo termo
\textit{handheld}, é considerado um computador de bolso por possuir interfaces
de entrada e saída de dados \textit{(input/output)}, uma tela \textit{(display)}
e um teclado numérico sucessivamente. Entretanto os dispositivos móveis
desenvolvidos atualmente como os \textit{palm-tops}, \textit{tablets} e
\textit{smartphones}, designados popularmente em inglês pela sigla PDA\textit{(
Personal Digital Assistent)}, diferenciam-se por proporcionar uma solução
integrada entre interface de entrada e de saída, a chamada interface
\textit{touchscreen}.

Estes dispositivos móveis atualmente, possuem soluções para múltiplas finalidades,
proporcionando ao usuário final comunicação através de serviços de voz e dados,
desde serviços mais simples como envio de mensagens SMS\textit{(Short Message
Service)} até a serviços mais complexos, como as webconferências, que exigem
conectividade com a \textit{internet} e demandam um intenso fluxo de dados para
disponibilização de serviços conhecidos como \textit{media streaming}.

Considerando essas potencialidades de uso e a diversificação de soluções em uma
escala de disponibilidade crescente, os \textit{smartphones} e \textit{tablets}
assumiram papel central nos últimos anos. Pesquisas recentes informam que o
número de usuários de internet móvel vem crescendo rapidamente, podendo superar
o número de usuários de internet por computadores pessoais ainda no ano de 2014
\cite{devitt2010meeker}. Estes números são muito expressivos, considerando a
crescente demanda de acesso à informação pela internet móvel a nível mundial.
% section contextualiza_o (end)
\section{Problemática} % (fold)
\label{sec:motiva_o}

Nos dias atuais, estamos vivenciando a chamada era da mobilidade, onde
dispositivos móveis estão cada vez mais presentes no cotidiano das pessoas, com
isso a diversificação existente no chamado ecossistema móvel, se destaca para
nós usuários finais nos sistemas operacionais existentes sejam eles Android,
iOS, Windows Phone e o novo Tizen.

(Problemática) Durante o processo de desenvolvimento de uma aplicação mobile o
fator inicial a ser levado em consideração é o sistema operacional a ser adotado.
Cada sistema operacional possui uma plataforma de desenvolvimento, ou seja, as
linguagens e os frameworks de desenvolvimento são distintos em cada plataforma,
essa diversificação torna-se um problema quando pensamos em portabilidade,
ou seja, desenvolver uma única aplicação que seja portável (usável) em
diferentes plataformas.
Muitas linguagens Android java, objective C, web, Mojo, .net
Estes aparelhos evoluíram e se diversificaram muito nos últimos anos, o que possibilitou a criação de
excelentes sistemas operacionais para eles. Alguns fabricantes, como a Apple e a Google, lançaram
ferramentas de desenvolvimento de software, para permitir que outras empresas e desenvolvedores
independentes possam desenvolver aplicativos para as respectivas plataformas.
Nesse sentido, a motivação desta pesquisa se baseia numa investigação das limitações existentes
acerca do desenvolvimento de aplicações mobile para as plataformas móveis. Pois, no contexto de alguns
sistemas operacionais móveis, este desenvolvimento pode se tornar uma atividade complexa desde a fase
inicial devido ao uso de ferramentas específicas e API’s (Interface de Programação de Aplicações) para
escrever o código em diferentes plataformas. Muitas vezes se torna complexo para os programadores
entenderem o que é preciso para desenvolver e distribuir uma determinada aplicação que implemente
serviços Web para um dispositivo específico. Como cada plataforma tem diferentes processos e requisitos
para a adesão destas aplicações, a documentação para partes diferentes do processo de desenvolvimento são
muitas vezes dispersas e difíceis de agrupar e sintetizar. Isto substancialmente caracteriza problemas
consideráveis quanto à interoperabilidade e convergência entre essas plataformas de desenvolvimento. Para
uma melhor compreensão deste estudo, este trabalho está dividido em duas contextualizações importantes: a
caracterização do chamado ecossistema móvel e as principais diferenças e limitações que a diversidade de
tecnologias impõe acerca do desenvolvimento de aplicações para dispositivos móveis.

Pretende-se, ao longo da pesquisa, descrever as características principais de um ecossistema móvel, as
relações existentes entre os sistemas operacionais móveis e o desenvolvimento de softwares para oss
mesmos, fazendo considerações tanto a respeito das aplicações nativas como das aplicações
multiplataforma, que proporcionam maior interoperabilidade ao usuário final. Portanto, serão abordadas as
principais características de algumas tecnologias (frameworks) importantes, que atualmente estão sendo
utilizadas nessa área de pesquisa, bem como serão também expostas algumas necessidades específicas que
podem contribuir para uma melhor convergência de soluções em softwares mobile, especialmente no que diz
respeito ao desenvolvimento multiplataforma, traçando parâmetros e requisitos segundo as recomendações
da W3C (World Wide Web Consortium).

Para solucionar esse problema, nossos estudos foram realizados no
chamado, desenvolvimento mobile multiplataforma, que utilizam frameworks que
proporcionam uma experiência transparente ao usuário final utilizando-se de
diferentes plataformas de desenvolvimento...

% section motiva_o (end)

\section{Objetivos da Pesquisa} % (fold)
\label{sec:objetivos}
Essa pesquisa propõe uma análise comparativa entre as duas principais ferramentas
de desenvolvimento móvel multiplataforma existentes atualmente, o
\textit{Adobe Phonegap} \textsuperscript{\texttrademark} e o
\textit{Appcelerator Titanium} \textsuperscript{\texttrademark}. Teremos como
objetivo, o desenvolvimento de uma solução móvel instável, de acordo com requisitos
pré-estabelecidos para uma aplicação multiplataforma. Estes requisitos são:
código aberto \textit{(open source)}, compatibilidade entre diferentes plataformas
móveis, acesso aos recursos de \textit{hardware}, comunicação
\textit{(server-side/back-end)}, interface de usuário \textit{(UI)} e segurança
de dados.
% section objetivos (end)

\section{Visão Geral} % (fold)
\label{sec:vis_o_geral}

% section vis_o_geral (end)
\section{Metodologia da Pesquisa} % (fold)
\label{sec:metodologia_de_pesquisa}
A metodologia aplicada a pesquisa foi subdividida em cinco capítulos. O
desenvolvimento está organizado de acordo com os seguintes conceitos:
\begin{itemize}
  \item O primeiro capítulo apresenta os conceitos referentes
    aos dispositivos móveis, os trabalhos relacionados e o objetivo da pesquisa.
  \item O segundo capítulo apresenta a potencial diversificação no
   desenvolvimento de tecnologias móveis e o conceito envolvendo a
   denominação ecossistema móvel, a diversificação existente em cada subdivisão
   e a sua influência no processo de engenharia e desenvolvimento de softwares
   para dispositivos móveis.
  \item No terceiro capítulo, será apresentado duas ferramentas para o
   desenvolvimento móvel multiplataforma, o \textit{PhoneGap} e o
   \textit{Titanium}. Apresentaremos as principais características de cada
   ferramenta, sua estrutura arquitetural, experiência de usuário
   \textit{(User Experience/UX)} e a interface de programação de aplicativos
   \textit{(API)}.
  \item No quarto capítulo, analisaremos critérios comparativos entre as
   duas ferramentas de desenvolvimento móvel multiplataforma como, a
   compatibilidade entre diferentes plataformas móveis, o acesso aos recursos de
   hardware e ao desempenho de cada ferramenta. A análise referente ao
   desempenho, será evidenciada de acordo com o desenvolvimento de uma
   aplicação que acessará diferentes recursos de \textit{hardware} para coleta de
   parâmetros referentes ao uso de memória, ao processamento de dados e ao
   consumo de energia.
  \item No quinto capítulo, será apresentada as considerações finais e as
   pesquisas futuras referentes a escolha de uma das ferramentas apresentadas,
   levando em consideração os parâmetros específicos de cada projeto de software.
\end{itemize}

É importante salientar que a pesquisa irá disponibilizar todo o código fonte
desenvolvido para a criação da aplicação utilizada como parâmetro comparativo
no quarto capítulo. O apêndice A contará com o código fonte utilizando o \textit{Apache Cordova} \textsuperscript{\texttrademark}, e o apêndice B com o código
fonte usando o \textit{Appcelerator Titanium} \textsuperscript{\texttrademark}.
% section metodologia_de_pesquisa (end)


% chapter chapter_name (end)