\chapter{Introdução} % (fold)
Ideias
(Contextualização) Citar pesquisas de fontes confiáveis para demonstrar a
popularizacao dos dispositivos móveis tanto no Brasil quando em Todo mundo,
seria bom também informar o crescimento e o futuro do crescimento...

(Motivação) Nos dias atuais, estamos vivenciando a chamada era da mobilidade, onde dispositivos
móveis estão cada vez mais presentes no cotidiano das pessoas, com isso a
diversificação existente no chamado ecossistema móvel, se destaca para nós,
usuários finais, nos sistemas operacionais existentes sejam eles Android, iOS,
Windows Phone e o novo Tizen.

(Problemática) Durante o processo de desenvolvimento de uma aplicação mobile o
fator inicial a ser levado em consideração é o sistema operacional a ser adotado.
Cada sistema operacional possui uma plataforma de desenvolvimento, ou seja, as
linguagens e os frameworks de desenvolvimento são distintos em cada plataforma,
essa diversificação torna-se um problema quando pensamos em portabilidade,
ou seja, desenvolver uma única aplicação que seja portável (usável) em
diferentes plataformas.

(Objetivos) Para solucionar esse problema, nossos estudos foram realizados no
chamado, desenvolvimento mobile multiplataforma, que utilizam frameworks que
proporcionam uma experiência transparente ao usuário final utilizando-se de
diferentes plataformas de desenvolvimento...

(Visão Geral dos capítulos)...
Lorem ipsum dolor sit amet, consectetur adipisicing elit, sed do eiusmod
tempor incididunt ut labore et dolore magna aliqua. Ut enim ad minim veniam,
quis nostrud exercitation ullamco laboris nisi ut aliquip ex ea commodo
consequat. Duis aute irure dolor in reprehenderit in voluptate velit esse
cillum dolore eu fugiat nulla pariatur. Excepteur sint occaecat cupidatat non
proident, sunt in culpa qui officia deserunt mollit anim id est laborum.

% chapter introducao (end)