
\documentclass{beamer}

\usepackage[brazil]{babel}
\usepackage[utf8]{inputenc}

\usepackage{lmodern}

\usetheme{Berlin}

\title{Análise comparativa no desenvolvimento mobile multiplataforma: PhoneGap vs. Titanium}
\author{Thuan Saraiva Nabuco}
\institute{Universidade Estadual do Ceará}
\date{\today}

\begin{document}
\titlepage
\begin{frame}
	\frametitle{Estrutura do Projeto}
	\framesubtitle{Desenvolvimento Mobile de aplicações híbridas}
	\begin{itemize}
		\item Tema de interesse
		\item Justificativa
		\item Problema
		\item Objetivos
		\item Trabalhos Relacionados
	\end{itemize}
\end{frame}

\begin{frame}
	\frametitle{Tema de Interesse}
	\framesubtitle{Apresente a justificativa do seu trabalho de pesquisa. Inicie com uma contextualização do trabalho e finalize com uma indicação de problema. Tente justificar por que a pesquisa está sendo realizada ou Por que é necessário pesquisar o tema que você indicou.}
	Analisar a perspectiva do desenvolvimento mobile multiplataforma, comparando dois dos mais usados frameworks de desenvolvimento, destacando características relevantes como: complexidade e compatibilidade com recursos nativos de cada plataforma de desenvolvimento.
\end{frame}

\begin{frame}
	\frametitle{Justificativa}
	\framesubtitle{}
	Durante o processo de desenvolvimento de uma aplicação mobile o fator inicial a ser levado em consideração é o sistema operacional a ser adotado. A diversificação, do chamado ecossistema mobile, se apresenta nos sistemas operacionais existentes, sejam eles Android, iOS ou Windows Phone. Cada sistema operacional possui uma plataforma de desenvolvimento, ou seja, as linguagens e os frameworks de desenvolvimento são distintos em cada plataforma, essa diversificação torna-se um problema quando pensamos em portabilidade, ou seja, desenvolver uma única aplicação que seja portável (usável) em diferentes plataformas. Para solucionar esses problemas, estudos estão sendo realizados no chamado desenvolvimento mobile multiplataforma que, se utilizam de frameworks que proporcionam uma experiência de portabilidade a diferentes plataformas de desenvolvimento.  

	que atinge não só os desenvolvedores mobile, que devem se especializar em mais de uma plataforma, mas até mesmo para empresas que pretendem disponibilizar um determinado (serviço/aplicação), que terá um custo a mais ao se optar por mais de uma plataforma de desenvolvimento tornando a decisão de se desenvolver uma única aplicação que será compatível com diferentes plataformas
\end{frame}

\begin{frame}
	\frametitle{Problema}
	\framesubtitle{Descreva o problema que você pretende tratar}
	
\end{frame}

\begin{frame}
	\frametitle{Objetivos}
	\framesubtitle{Objetivos Gerais}
	O que você irá propor na sua pesquisa? Qual o objetivo maior do seu trabalho ?
\end{frame}

\begin{frame}
	\frametitle{Objetivos}
	\framesubtitle{Objetivos Específicos}
	Quais os objetivos menores a serem alcançados ?
\end{frame}

\begin{frame}[t]
	\frametitle{Trabalhos Relacionados}
	\framesubtitle{Descrever Trabalhos Relacionados que você encontrou sobre o assunto}
	\begin{itemize}
			\item \href{http://ieeexplore.ieee.org/stamp/stamp.jsp?arnumber=6583580} {IEEE - Survey, Comparison and Evaluation of Cross Platform Mobile Application Development Tools};		
			\item \href {http://ieeexplore.ieee.org/stamp/stamp.jsp?arnumber=6376023}{IEEE - Comparison of Cross-Platform Mobile Development Tools};
			\item \href{http://ieeexplore.ieee.org/stamp/stamp.jsp?arnumber=6549324}{IEEE - Portability of Mobile Applications using PhoneGap: A case Study};
			\item \href{http://www.nngroup.com/articles/mobile-native-apps/}{Artigo: Comparação entre Apps Mobile Nativo, Híbrido e Web};
			\item \href{https://www.ibm.com/developerworks/community/blogs/ctaurion/entry/desenvolvimento_de_apps-parte_2_hibrido_nativo_ou_web?lang=en}{Artigo: Comparação entre Apps Mobile Nativo, Híbrido e Web - IBM};
			\item \href{http://luisaambros.com/blog/diferenca-entre-aplicativos-nativos-hibridos-e-mobile-web-apps/}{Diferença entre app's nativos híbridos e Mobile WebApps - Luisa Ambros}; 
		\end{itemize}	
\end{frame}

\begin{frame}
	\frametitle{Trabalhos Relacionados}
	\framesubtitle{Descrever Trabalhos Relacionados que você encontrou sobre o assunto}
	\begin{itemize}
			\item Um artigo que realmente trata do assunto que me interessou, nele foi apresentada uma pesquisa sobre o chamado WORA (Write one run anywhere) além de exemplificar uma única aplicação feita em cada uma das tecnologias (Phonegap, Sencha, jQueryMobile, Titanium).
			\item Nesse artigo o foco além de abordar o conceito de Portabilidade Mobile também expõe anpalises sobre um dos frameworks multiplataforma, no caso o PhoneGap.
			\item Esses links foram referências para desenvolvimento inicial da pesquisa sobre a real justificativa para se pesquisar sobre o Desenvolvimento Mobile Multiplataforma, seja ela Híbrida ou Web.
		\end{itemize}	
	
\end{frame}

\end{document}