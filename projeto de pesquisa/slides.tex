
\documentclass{beamer}

\usepackage[brazil]{babel}
\usepackage[utf8]{inputenc}
\usepackage{lmodern}

\usetheme{Berlin}

\title{Análise comparativa no desenvolvimento mobile multiplataforma: PhoneGap vs. Titanium}
\author{Thuan Saraiva Nabuco}
\institute{Universidade Estadual do Ceará}
\date{\today}

\begin{document}
\titlepage
\begin{frame}
	\frametitle{Estrutura do Projeto}
	\framesubtitle{Desenvolvimento Mobile de aplicações híbridas}
	\begin{itemize}
		\item Tema de interesse
		\item Justificativa
		\item Problema
		\item Objetivos
		\item Trabalhos Relacionados
	\end{itemize}
\end{frame}

\begin{frame}
	\frametitle{Tema de Interesse}
	\framesubtitle{}
	Analisar a perspectiva do desenvolvimento mobile multiplataforma, comparando dois dos mais usados frameworks de desenvolvimento, destacando características relevantes como: complexidade e compatibilidade com recursos nativos de cada plataforma de desenvolvimento.
\end{frame}

\begin{frame}
	\frametitle{Justificativa}
	\framesubtitle{Apresente a justificativa do seu trabalho de pesquisa. Inicie com uma contextualização do trabalho e finalize com uma indicação de problema. Tente justificar por que a pesquisa está sendo realizada ou Por que é necessário pesquisar o tema que você indicou.}
	Ao iniciar o processo de desenvolvimento de uma aplicação mobile o primeiro fator a ser levado em consideração é o sistema operacional a ser adotado. A diversificação do chamado ecossistema mobile se apresenta em divesos sistemas operacionais existentes sejam eles Android, iOS ou Windows Phone. a decisão de se desenvolver uma única aplicação que será compatível com diferentes plataformas
\end{frame}

\begin{frame}
	\frametitle{Problema}
	\framesubtitle{Descreva o problema que você pretende tratar}
	
\end{frame}

\begin{frame}
	\frametitle{Objetivos}
	\framesubtitle{Objetivos Gerais}
	O que você irá propor na sua pesquisa? Qual o objetivo maior do seu trabalho ?
\end{frame}

\begin{frame}
	\frametitle{Objetivos}
	\framesubtitle{Objetivos Específicos}
	Quais os objetivos menores a serem alcançados ?
\end{frame}

\begin{frame}
	\frametitle{Trabalhos Relacionados}
	\framesubtitle{Descrever Trabalhos Relacionados que você encontrou sobre o assunto}
	
\end{frame}


\end{document}