
\documentclass[10pt]{beamer}

\usepackage[brazil]{babel}
\usepackage[utf8]{inputenc}

\usepackage{lmodern}

\usetheme{Berlin}

\title{Análise comparativa no desenvolvimento mobile multiplataforma: PhoneGap vs. Titanium}
\author{Thuan Saraiva Nabuco}
\institute{Universidade Estadual do Ceará}
\date{\today}

\begin{document}
\titlepage
\begin{frame}
	\frametitle{Estrutura do Projeto}
	\framesubtitle{Desenvolvimento Mobile de aplicações híbridas}
	\begin{itemize}
		\item Tema de interesse
		\item Justificativa
		\item Problema
		\item Objetivos
		\item Trabalhos Relacionados
	\end{itemize}
\end{frame}

\begin{frame}
	\frametitle{Tema de Interesse}
	\framesubtitle{Área de pesquisa}
	Analisar a perspectiva do desenvolvimento mobile multiplataforma, comparando dois dos mais usados frameworks de desenvolvimento, destacando características relevantes como: complexidade e compatibilidade com recursos nativos em cada plataforma de desenvolvimento.
\end{frame}

\begin{frame}[t]
	\frametitle{Justificativa}
	\framesubtitle{Contextualização, justificativa e indicação do problema}
	 A diversificação do chamado ecossistema mobile se destaca para nós, usuários finais, nos sistemas operacionais existentes sejam eles Android, iOS, Windows Phone e o novo Tizen. Durante o processo de desenvolvimento de uma aplicação mobile o fator inicial a ser levado em consideração é o sistema operacional a ser adotado. Cada sistema operacional possui uma plataforma de desenvolvimento, ou seja, as linguagens e os frameworks de desenvolvimento são distintos em cada plataforma, essa diversificação torna-se um problema quando pensamos em portabilidade, ou seja, desenvolver uma única aplicação que seja portável (usável) em diferentes plataformas. Para solucionar esse problema, nossos estudos foram realizados no chamado, desenvolvimento mobile multiplataforma, que utilizam frameworks que proporcionam uma experiência transparente ao usuário final utilizando-se de diferentes plataformas de desenvolvimento. 
\end{frame}

\begin{frame}
	\frametitle{Problema}
	\framesubtitle{Descreva o problema que você pretende tratar}
	O problema da diversificação existente nos sistemas operacionais móveis não atige somente aos desenvolvedores independentes mas também atigem as empresas que pretendem disponibilizar um determinado aplicativo e/ou serviço e não contam com equipes de desenvolvimento especializadas em cada plataforma, o que viabiliza um custo adicional ao processo de desenvolvimento.
\end{frame}

\begin{frame}
	\frametitle{Objetivos}
	\framesubtitle{Objetivos Gerais}
	Através da comparação entre PhoneGap e Titanium, temos como principal objetivo expor características relevantes ao processo de desenvolvimento no momento da adoção pela abordagem de desenvolvimento mobile multiplataforma.
\end{frame}

\begin{frame}
	\frametitle{Objetivos}
	\framesubtitle{Objetivos Específicos}
	\begin{itemize}
		\item Expor prós e contras de cada tecnologia;
		\item Enfatizar a importância de se analisar o contexto, levando em consideração requisitos funcionais e não-funcionais da aplicação;
		\item Prover uma pesquisa que exponha o avanço das tecnologias alternativas ao desenvolvimento mobile nativo;
	\end{itemize}
\end{frame}

\begin{frame}
	\frametitle{Trabalhos Relacionados}
	\framesubtitle{Trabalhos Relacionados sobre o assunto}
	\begin{itemize}
			\item \href{http://ieeexplore.ieee.org/stamp/stamp.jsp?arnumber=6583580} {IEEE - Survey, Comparison and Evaluation of Cross Platform Mobile Application Development Tools};		
			\item \href{http://ieeexplore.ieee.org/stamp/stamp.jsp?arnumber=6376023}{IEEE - Comparison of Cross-Platform Mobile Development Tools};
			\item \href{http://ieeexplore.ieee.org/stamp/stamp.jsp?arnumber=6549324}{IEEE - Portability of Mobile Applications using PhoneGap: A case Study};
			\item \href{http://www.nngroup.com/articles/mobile-native-apps/}{Link: Comparação entre Apps Mobile Nativo, Híbrido e Web};
			\item \href{https://www.ibm.com/developerworks/community/blogs/ctaurion/entry/desenvolvimento_de_apps-parte_2_hibrido_nativo_ou_web?lang=en}{Link: Comparação entre Apps Mobile Nativo, Híbrido e Web - IBM};
			\item \href{http://luisaambros.com/blog/diferenca-entre-aplicativos-nativos-hibridos-e-mobile-web-apps/}{Link: Diferença entre app's nativos híbridos e Mobile WebApps - Luisa Ambros}; 
	\end{itemize}	
\end{frame}

\begin{frame}
	\frametitle{Trabalhos Relacionados}
	\framesubtitle{Descrição de Trabalhos Relacionados sobre o assunto}
	\begin{itemize}
			\item No primeiro artigo IEEE - Survey, Comparison and Evaluation of Cross Platform Mobile Application Development Tools, os autores centralizam as suas comparações justamente no PhoneGap e no Titanium, e avaliam a performance de cada um levando em consideração o uso de memória, uso da CPU e o consumo de energia dos dispositivos.
			\item No segundo Artigo IEEE - Comparison of Cross-Platform Mobile Development Tools, os autores enfatizam além do PhoneGap, o Rhodes e sua arquitetura MVC, exclusiva entre as ferramentas de desenvolvimento multiplataforma.
			\item No terceiro artigo IEEE - Portability of Mobile Applications using PhoneGap: A case Study, os autores possuem um problema, ou seja uma aplicação que precisaria ser disponivel em mais de uma plataforma mobile, avaliou os diversos frameworks de desenvolvimento optando pelo PhoneGap e realizando um estudo de caso baseado nessa tecnologia.
			\item Esses links foram referências para desenvolvimento inicial da pesquisa sobre a real justificativa para se pesquisar sobre o Desenvolvimento Mobile Multiplataforma, seja ela Híbrida ou Web.
	\end{itemize}	
\end{frame}

\end{document}