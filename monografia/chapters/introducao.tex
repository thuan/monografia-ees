\chapter{Introdução} % (fold)
\label{cha:introducao}

Nos dias atuais, estamos presenciando a chamada era da mobilidade, onde dispositivos móveis estão cada vez mais presentes no cotidiano das pessoas, com
isso  A diversificação existente no chamado ecossistema móvel,
A diversificação do chamado ecossistema mobile se destaca para nós, usuários finais, nos sistemas operacionais existentes sejam eles Android, iOS, Windows Phone e o novo Tizen. Durante o processo de desenvolvimento de uma aplicação mobile o fator inicial a ser levado em consideração é o sistema operacional a ser adotado. Cada sistema operacional possui uma plataforma de desenvolvimento, ou seja, as linguagens e os frameworks de desenvolvimento são distintos em cada plataforma, essa diversificação torna-se um problema quando pensamos em portabilidade, ou seja, desenvolver uma única aplicação que seja portável (usável) em diferentes plataformas. Para solucionar esse problema, nossos estudos foram realizados no chamado, desenvolvimento mobile multiplataforma, que utilizam frameworks que proporcionam uma experiência transparente ao usuário final utilizando-se de diferentes plataformas de desenvolvimento.
% chapter introducao (end)

\chapter{chapter name} % (fold)
\label{cha:chapter_name}

% chapter chapter_name (end)