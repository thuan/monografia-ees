% v0.4 - 2011071901
%
% Modelo de monografia em LaTeX da UECE baseado no template disponibilizado
% por Rudy Matela em http://www.larces.uece.br/~rudy/pub/modelo_monografia/

\input{modelo/tex/preambulo}


% Informações gerais do documento
\autor{Thuan Saraiva Nabuco}
\autorr{Nabuco, Thuan Saraiva}
\titulo{O Desenvolvimento móvel multiplataforma: Uma análise
comparativa entre \textit{Adobe PhoneGap}
\textsuperscript{\textregistered} e \textit{Appcelerator Titanium} \textsuperscript{\textregistered}}
\orientador{Paulo Henrique Mendes Maia}

% essas informacoes do codigo CIP você consegue indo na biblioteca central
\codigocip{C824p}{CDD:001.6}


% Epígrafe: citação e autor (OPCIONAL)
\epigrafe{``Conhecimento não é aquilo que você sabe, mas o que você faz com
aquilo que você sabe. ''}
\autorepigrafe{Aldous Huxley}


% Membros da comissão avaliadora
\bancaum{Prof. Dr. \ABNTorientadordata\\
Universidade Estadual do Ceará -- UECE\\Orientador}
\bancadois{Prof. Dr. Cidcley Teixeira de Souza\\
Instituto Federal do Ceará -- IFCE}
\bancatres{Prof. MSc. Március Gomes Brandão\\
Universidade Estadual do Ceará -- UECE}

% data de aprovação
\dataaprovacao{15/07/2014}

% Agradecimentos
\agradecimentostext{\noindent
À minha família, pelo incentivo e apoio nos momentos difíceis.

\noindent
Aos amigos que conheci na UECE ao longo destes anos, pela ajuda e
amizade.

\noindent
Ao professor Paulo Henrique Mendes Maia pela orientação e oportunidades de
iniciação à pesquisa.

\noindent
À maioria dos professores que tive durante este curso de Especialização
em Engenharia de Software com ênfase em padrões de software da UECE, por sua
dedicação, comprometimento e compreensão.

\noindent
A todas as pessoas que passaram pela minha vida e contribuíram para a
construção de quem sou hoje.

}

% Resumo/palavras chaves
\resumotext{O objetivo deste trabalho é apresentar as principais características
do desenvolvimento móvel multiplataforma, comparando duas ferramentas que irão
auxiliar o processo de desenvolvimento de soluções que sejam compatíveis com os
requisitos obrigatórios para que uma aplicação seja portável em diversas
plataformas, ou seja, seja cosiderada uma aplicação multiplataforma. A pesquisa
é pautada diante da evolução e consequentemente da diversificação existente nas
tecnologias móveis, caracterizando uma denominação conhecida como ecossistema móvel.
A metodologia aplicada .Os resultados .
}
\pcs{Desenvolvimento multiplataforma}{Desenvolvimento móvel}{Phonegap}{Titanium}

% abstract/keywords
\abstracttext{VANETs (Vehicular Ad hoc NETworks) are a special type of the
Mobile Ad hoc NETworks (MANETs), made by vehicles communicating between
themselves as well as by vehicles communicating with devices located at the
margins of roads and highways. Despite sharing many characteristics with the
traditional MANETs, VANETs present some significant differences. For instance,
the nodes movement, completely random in MANETs, but relatively ordered in the
VANETs, since the nodes -- the vehicles -- are supposed to obey a set of
transit rules. In this work, the bio-inspired routing protocol Ant-DYMO is
evaluated in a VANET scenario and has its performance compared with the DYMO
protocol. Furthermore, a simple modification is proposed in the bio-inspired
mechanisms of Ant-DYMO. The objective is to verify a possible improvement in
the overall performance of the algorithm when taking into account some
characteristics inherent to the vehicular networks, such as neighboring
vehicle information.}
\kws{Cross-platform development}{Mobile development}{Phonegap}{Titanium}



%%%%%%%%%%%%%%%%%%%%%%%%%%%%%%%%%%%%%%%%%%%%%%%%%%%%%%%%%%%%%%%%%%%%%%%%%%%%
% inicio do documento
%%%%%%%%%%%%%%%%%%%%%%%%%%%%%%%%%%%%%%%%%%%%%%%%%%%%%%%%%%%%%%%%%%%%%%%%%%%%

\begin{document}
% o arquivo seguinte contém os elementos pre-textuais. edite-o caso deseje
% adicionar/remover algum elemento.
\input{modelo/tex/documento}

% agora adicione os capítulos
\chapter{Introdução} % (fold)
Devido a constante evolução das redes de comunicação e ao surgimento continuo de
novas tecnologias, as tecnologias móveis tornaram-se cada vez mais importantes e
presentes no cotidiano das pessoas, sendo utilizadas em pesquisas avançadas no
meio acadêmico, corporativo, no entretenimento e no auxilio as atividades
pessoais dos seus usuários. Novas vertentes relacionadas a usabilidade
destas tecnologias móveis surgiram principalmente pela evolução das redes móveis,
algo que está sendo viabilizado principalmente pela crescente popularização e
diversificação dos dispositivos móveis.

Um dispositivo móvel, designado popularmente em inglês por \textit{handheld}, é
considerado um computador de bolso por possuir interfaces de entrada e saída
de dados, \textit{(input/output)}, uma tela \textit{(display)} e um teclado
especificamente. Entretanto os dispositivos móveis como \textit{palm-tops}, \textit{tablets}
e \textit{smartphones} provem uma solução integrada entre estas duas interfaces,
a interface \textit{touch screen}.

Os dispositivos móveis hoje possuem soluções para múltiplas finalidades, proporcionando
ao usuário final comunicação através de serviços de voz e dados, desde serviços
mais simples como envio de mensagens SMS\textit{(Short Message Service)}
até a serviços mais complexos, como as webconferências, que e exigem conectividade
com a \textit{internet} e demandam um intenso fluxo de dados para
disponibilização de serviços conhecidos como \textit{media streaming}.

Considerando estas potencialidades de uso e à diversificação de soluções em uma
escala de disponibilidade crescente, os \textit{smartphones} e \textit{tablets}
assumiram papel central nos últimos anos. Dados da Anatel indicam que o Brasil
terminou o ano de 2012 com 242,2 milhões de dispositivos móveis conectáveis
a uma densidade de 123,9 disp/100 hab. Estes números são muito expressivos,
principalmente considerando a crescente demanda de acesso à informação pela internet móvel.

Estes aparelhos evoluíram e se diversificaram muito nos últimos anos, o que possibilitou a criação de
excelentes sistemas operacionais para eles. Alguns fabricantes, como a Apple e a Google, lançaram
ferramentas de desenvolvimento de software, para permitir que outras empresas e desenvolvedores
independentes possam desenvolver aplicativos para as respectivas plataformas.
Nesse sentido, a motivação desta pesquisa se baseia numa investigação das limitações existentes
acerca do desenvolvimento de aplicações mobile para as plataformas móveis. Pois, no contexto de alguns
sistemas operacionais móveis, este desenvolvimento pode se tornar uma atividade complexa desde a fase
inicial devido ao uso de ferramentas específicas e API’s (Interface de Programação de Aplicações) para
escrever o código em diferentes plataformas. Muitas vezes se torna complexo para os programadores
entenderem o que é preciso para desenvolver e distribuir uma determinada aplicação que implemente
serviços Web para um dispositivo específico. Como cada plataforma tem diferentes processos e requisitos
para a adesão destas aplicações, a documentação para partes diferentes do processo de desenvolvimento são
muitas vezes dispersas e difíceis de agrupar e sintetizar. Isto substancialmente caracteriza problemas
consideráveis quanto à interoperabilidade e convergência entre essas plataformas de desenvolvimento. Para
uma melhor compreensão deste estudo, este trabalho está dividido em duas contextualizações importantes: a
caracterização do chamado ecossistema móvel e as principais diferenças e limitações que a diversidade de
tecnologias impõe acerca do desenvolvimento de aplicações para dispositivos móveis.

Pretende-se, ao longo da pesquisa, descrever as características principais de um ecossistema móvel, as
relações existentes entre os sistemas operacionais móveis e o desenvolvimento de softwares para oss
mesmos, fazendo considerações tanto a respeito das aplicações nativas como das aplicações
multiplataforma, que proporcionam maior interoperabilidade ao usuário final. Portanto, serão abordadas as
principais características de algumas tecnologias (frameworks) importantes, que atualmente estão sendo
utilizadas nessa área de pesquisa, bem como serão também expostas algumas necessidades específicas que
podem contribuir para uma melhor convergência de soluções em softwares mobile, especialmente no que diz
respeito ao desenvolvimento multiplataforma, traçando parâmetros e requisitos segundo as recomendações
da W3C (World Wide Web Consortium).




(Motivação) Nos dias atuais, estamos vivenciando a chamada era da mobilidade, onde dispositivos
móveis estão cada vez mais presentes no cotidiano das pessoas, com isso a
diversificação existente no chamado ecossistema móvel, se destaca para nós,
usuários finais, nos sistemas operacionais existentes sejam eles Android, iOS,
Windows Phone e o novo Tizen. \cite{adriel2012titanium}

(Problemática) Durante o processo de desenvolvimento de uma aplicação mobile o
fator inicial a ser levado em consideração é o sistema operacional a ser adotado.
Cada sistema operacional possui uma plataforma de desenvolvimento, ou seja, as
linguagens e os frameworks de desenvolvimento são distintos em cada plataforma,
essa diversificação torna-se um problema quando pensamos em portabilidade,
ou seja, desenvolver uma única aplicação que seja portável (usável) em
diferentes plataformas.
Muitas linguagens Android java, objective C, web, Mojo, .net

(Objetivos) Para solucionar esse problema, nossos estudos foram realizados no
chamado, desenvolvimento mobile multiplataforma, que utilizam frameworks que
proporcionam uma experiência transparente ao usuário final utilizando-se de
diferentes plataformas de desenvolvimento...

(Visão Geral dos capítulos)...
Lorem ipsum dolor sit amet, consectetur adipisicing elit, sed do eiusmod
tempor incididunt ut labore et dolore magna aliqua. Ut enim ad minim veniam,
quis nostrud exercitation ullamco laboris nisi ut aliquip ex ea commodo
consequat. Duis aute irure dolor in reprehenderit in voluptate velit esse
cillum dolore eu fugiat nulla pariatur. Excepteur sint occaecat cupidatat non
proident, sunt in culpa qui officia deserunt mollit anim id est laborum.


% chapter introducao (end)
\chapter{Referencial Teórico} % (fold)
Este capítulo tem por objetivo enunciar a potencial diversificação no
desenvolvimento de tecnologias móveis. Para isso, iremos apresentar o conceito
envolvendo a denominação ecossistema móvel, a diversificação existente
em cada subdivisão de um ecossistema móvel e o que isto influenciará diretamente
no processo de engenharia e desenvolvimento de softwares para dispositivos móveis.


\section{Ecossistema Móvel} % (fold)
(Idéia Central) Essa seção terá objetivo explicar o significado do chamado
``Ecossistema móvel'' e exemplificar a diversidade existente nos dias atuais
o que proporciona o surgimento do nosso objeto de estudo que no caso será o
desenvolvimento móvel multiplataforma.

Exemplificaremos as estruturas subdivididas em camadas, figuras representativas
serão inseridas nessa seção. Acredito que utilizaremos no máximo 3 parágrafos e 2
imagens explicativas.

Lorem ipsum dolor sit amet, consectetur adipisicing elit, sed do eiusmod
tempor incididunt ut labore et dolore magna aliqua. Ut enim ad minim veniam,
quis nostrud exercitation ullamco laboris nisi ut aliquip ex ea commodo
consequat. Duis aute irure dolor in reprehenderit in voluptate velit esse
cillum dolore eu fugiat nulla pariatur. Excepteur sint occaecat cupidatat non
proident, sunt in culpa qui officia deserunt mollit anim id est laborum.

Lorem ipsum dolor sit amet, consectetur adipisicing elit, sed do eiusmod
tempor incididunt ut labore et dolore magna aliqua. Ut enim ad minim veniam,
quis nostrud exercitation ullamco laboris nisi ut aliquip ex ea commodo
consequat. Duis aute irure dolor in reprehenderit in voluptate velit esse
cillum dolore eu fugiat nulla pariatur. Excepteur sint occaecat cupidatat non
proident, sunt in culpa qui officia deserunt mollit anim id est laborum.
% section o_desenvolvimento_multiplataforma (end)

\section{Desenvolvimento móvel} % (fold)
Definições iniciais a respeito do desenvolvimento móvel, desenvolvimento nativo
(exemplificar iOS - objectiveC, Android- Java) desenvolvimento web móvel (HTML5)

\subsection{Plataformas nativas} % (fold)

Lorem ipsum dolor sit amet, consectetur adipisicing elit, sed do eiusmod
tempor incididunt ut labore et dolore magna aliqua. Ut enim ad minim veniam,
quis nostrud exercitation ullamco laboris nisi ut aliquip ex ea commodo
consequat. Duis aute irure dolor in reprehenderit in voluptate velit esse
cillum dolore eu fugiat nulla pariatur. Excepteur sint occaecat cupidatat non
proident, sunt in culpa qui officia deserunt mollit anim id est laborum.
% subsection plataformas_nativas (end)

\subsection{Web} % (fold)
Plataforma Web características
Lorem ipsum dolor sit amet, consectetur adipisicing elit, sed do eiusmod
tempor incididunt ut labore et dolore magna aliqua. Ut enim ad minim veniam,
quis nostrud exercitation ullamco laboris nisi ut aliquip ex ea commodo
consequat. Duis aute irure dolor in reprehenderit in voluptate velit esse
cillum dolore eu fugiat nulla pariatur. Excepteur sint occaecat cupidatat non
proident, sunt in culpa qui officia deserunt mollit anim id est laborum.
% subsection web (end)

\section{Desenvolvimento móvel multiplataforma} % (fold)

Definição sobre o desenvolvimento móvel multiplataforma, citando artigos, livros e as
teses existentes, no máximo 3 parágrafos.

Lorem ipsum dolor sit amet, consectetur adipisicing elit, sed do eiusmod
tempor incididunt ut labore et dolore magna aliqua. Ut enim ad minim veniam,
quis nostrud exercitation ullamco laboris nisi ut aliquip ex ea commodo
consequat. Duis aute irure dolor in reprehenderit in voluptate velit esse
cillum dolore eu fugiat nulla pariatur. Excepteur sint occaecat cupidatat non
proident, sunt in culpa qui officia deserunt mollit anim id est laborum.
% section desenvolvimento_móvel_multiplataforma (end)

\subsection{Requisitos} % (fold)
(IEEE - Survey, Comparison and Evaluation)
Esse tópico irá informar os requisitos desejáveis para um framework de desenvolvimento móvel multiplataforma, 1 parágrafo:
\begin{enumerate}
	\item[$\bullet$] Múltiplas plataformas móveis:

	Lorem ipsum dolor sit amet, consectetur adipisicing elit, sed do eiusmod
	tempor incididunt ut labore et dolore magna aliqua. Ut enim ad minim veniam,
	quis nostrud exercitation ullamco laboris nisi ut aliquip ex ea commodo
	consequat. Duis aute irure dolor in reprehenderit in voluptate velit esse
	cillum dolore eu fugiat nulla pariatur. Excepteur sint occaecat cupidatat non
	proident, sunt in culpa qui officia deserunt mollit anim id est laborum.

	\item[$\bullet$] Interface de usuário \textit{(Rich User Interface)}:

	Lorem ipsum dolor sit amet, consectetur adipisicing elit, sed do eiusmod
	tempor incididunt ut labore et dolore magna aliqua. Ut enim ad minim veniam,
	quis nostrud exercitation ullamco laboris nisi ut aliquip ex ea commodo
	consequat. Duis aute irure dolor in reprehenderit in voluptate velit esse
	cillum dolore eu fugiat nulla pariatur. Excepteur sint occaecat cupidatat non
	proident, sunt in culpa qui officia deserunt mollit anim id est laborum.

	\item[$\bullet$] Segurança:

	Lorem ipsum dolor sit amet, consectetur adipisicing elit, sed do eiusmod
	tempor incididunt ut labore et dolore magna aliqua. Ut enim ad minim veniam,
	quis nostrud exercitation ullamco laboris nisi ut aliquip ex ea commodo
	consequat. Duis aute irure dolor in reprehenderit in voluptate velit esse
	cillum dolore eu fugiat nulla pariatur. Excepteur sint occaecat cupidatat non
	proident, sunt in culpa qui officia deserunt mollit anim id est laborum.

	\item[$\bullet$] Comunicação \textit{Back-end}:

	Lorem ipsum dolor sit amet, consectetur adipisicing elit, sed do eiusmod
	tempor incididunt ut labore et dolore magna aliqua. Ut enim ad minim veniam,
	quis nostrud exercitation ullamco laboris nisi ut aliquip ex ea commodo
	consequat. Duis aute irure dolor in reprehenderit in voluptate velit esse
	cillum dolore eu fugiat nulla pariatur. Excepteur sint occaecat cupidatat non
	proident, sunt in culpa qui officia deserunt mollit anim id est laborum.

	\item[$\bullet$] Acesso aos recursos internos:

	Lorem ipsum dolor sit amet, consectetur adipisicing elit, sed do eiusmod
	tempor incididunt ut labore et dolore magna aliqua. Ut enim ad minim veniam,
	quis nostrud exercitation ullamco laboris nisi ut aliquip ex ea commodo
	consequat. Duis aute irure dolor in reprehenderit in voluptate velit esse
	cillum dolore eu fugiat nulla pariatur. Excepteur sint occaecat cupidatat non
	proident, sunt in culpa qui officia deserunt mollit anim id est laborum.

	\item[$\bullet$] Código aberto \textit{(Open Source)}:

	Lorem ipsum dolor sit amet, consectetur adipisicing elit, sed do eiusmod
	tempor incididunt ut labore et dolore magna aliqua. Ut enim ad minim veniam,
	quis nostrud exercitation ullamco laboris nisi ut aliquip ex ea commodo
	consequat. Duis aute irure dolor in reprehenderit in voluptate velit esse
	cillum dolore eu fugiat nulla pariatur. Excepteur sint occaecat cupidatat non
	proident, sunt in culpa qui officia deserunt mollit anim id est laborum.
\end{enumerate}

Lorem ipsum dolor sit amet, consectetur adipisicing elit, sed do eiusmod
tempor incididunt ut labore et dolore magna aliqua. Ut enim ad minim veniam,
quis nostrud exercitation ullamco laboris nisi ut aliquip ex ea commodo
consequat. Duis aute irure dolor in reprehenderit in voluptate velit esse
cillum dolore eu fugiat nulla pariatur. Excepteur sint occaecat cupidatat non
proident, sunt in culpa qui officia deserunt mollit anim id est laborum.

% subsection requisitos_para_uma_ferramenta_de_desenvolvimento_móvel_multiplataforma (end)

\subsection{Ferramentas} % (fold)
(IEEE - Survey, Comparison and Evaluation)

Lorem ipsum dolor sit amet, consectetur adipisicing elit, sed do eiusmod
tempor incididunt ut labore et dolore magna aliqua. Ut enim ad minim veniam,
quis nostrud exercitation ullamco laboris nisi ut aliquip ex ea commodo
consequat. Duis aute irure dolor in reprehenderit in voluptate velit esse
cillum dolore eu fugiat nulla pariatur. Excepteur sint occaecat cupidatat non
proident, sunt in culpa qui officia deserunt mollit anim id est laborum.


\subsection{Arquitetura} % (fold)
(IEEE - Survey, Comparison and Evaluation)
Lorem ipsum dolor sit amet, consectetur adipisicing elit, sed do eiusmod
tempor incididunt ut labore et dolore magna aliqua. Ut enim ad minim veniam,
quis nostrud exercitation ullamco laboris nisi ut aliquip ex ea commodo
consequat. Duis aute irure dolor in reprehenderit in voluptate velit esse
cillum dolore eu fugiat (Figura \ref{fig:arch}) nulla pariatur. Excepteur sint occaecat cupidatat non
proident, sunt in culpa qui officia deserunt mollit anim id est laborum.

\begin{figure}[htbp]
\centering
 \includegraphics[width=.65\textwidth]{chapters/fig/archGeneral.pdf}
\caption{Arquitetura para o desenvolvimento de aplicações móveis multiplataforma}
\label{fig:arch}
\end{figure}

Lorem ipsum dolor sit amet, consectetur adipisicing elit, sed do eiusmod
tempor incididunt ut labore et dolore magna aliqua. Ut enim ad minim veniam,
quis nostrud exercitation ullamco laboris nisi ut aliquip ex ea commodo
consequat. Duis aute irure dolor in reprehenderit in voluptate velit esse
cillum dolore eu fugiat nulla pariatur. Excepteur sint occaecat cupidatat non
proident, sunt in culpa qui officia deserunt mollit anim id est laborum.
% subsection arquitetura_geral (end)

% chapter desenvolvimento_móvel_multiplataforma (end)
\chapter{Ferramentas para o desenvolvimento móvel multiplataforma}

\section{Adobe PhoneGap} % (fold)
Lorem ipsum dolor sit amet, consectetur adipisicing elit, sed do eiusmod
tempor incididunt ut labore et dolore magna aliqua. Ut enim ad minim veniam,
quis nostrud exercitation ullamco laboris nisi ut aliquip ex ea commodo
consequat. Duis aute irure dolor in reprehenderit in voluptate velit esse
cillum dolore eu fugiat nulla pariatur. Excepteur sint occaecat cupidatat non
proident, sunt in culpa qui officia deserunt mollit anim id est laborum.

Lorem ipsum dolor sit amet, consectetur adipisicing elit, sed do eiusmod
tempor incididunt ut labore et dolore magna aliqua. Ut enim ad minim veniam,
quis nostrud exercitation ullamco laboris nisi ut aliquip ex ea commodo
consequat. Duis aute irure dolor in reprehenderit in voluptate velit esse
cillum dolore eu fugiat nulla pariatur. Excepteur sint occaecat cupidatat non
proident, sunt in culpa qui officia deserunt mollit anim id est laborum.

\subsection{Estrutura Arquitetural} % (fold)
Lorem ipsum dolor sit amet, consectetur adipisicing elit, sed do eiusmod
tempor incididunt ut labore et dolore magna aliqua. Ut enim ad minim veniam,
quis nostrud exercitation ullamco laboris nisi ut aliquip ex ea commodo
consequat. Duis aute irure dolor in reprehenderit in voluptate velit esse
cillum dolore eu fugiat nulla pariatur. Excepteur sint occaecat cupidatat non
proident, sunt in culpa qui officia deserunt mollit anim id est laborum.

Lorem ipsum dolor sit amet, consectetur adipisicing elit, sed do eiusmod
tempor incididunt ut labore et dolore magna aliqua. Ut enim ad minim veniam,
quis nostrud exercitation ullamco laboris nisi ut aliquip ex ea commodo
consequat. Duis aute irure dolor in reprehenderit in voluptate velit esse
cillum dolore eu fugiat nulla pariatur. Excepteur sint occaecat cupidatat non
proident, sunt in culpa qui officia deserunt mollit anim id est laborum.
% subsection estrutura_arquitetural (end)
\subsection{Experiência de usuário} % (fold)
Lorem ipsum dolor sit amet, consectetur adipisicing elit, sed do eiusmod
tempor incididunt ut labore et dolore magna aliqua. Ut enim ad minim veniam,
quis nostrud exercitation ullamco laboris nisi ut aliquip ex ea commodo
consequat. Duis aute irure dolor in reprehenderit in voluptate velit esse
cillum dolore eu fugiat nulla pariatur. Excepteur sint occaecat cupidatat non
proident, sunt in culpa qui officia deserunt mollit anim id est laborum.

Lorem ipsum dolor sit amet, consectetur adipisicing elit, sed do eiusmod
tempor incididunt ut labore et dolore magna aliqua. Ut enim ad minim veniam,
quis nostrud exercitation ullamco laboris nisi ut aliquip ex ea commodo
consequat. Duis aute irure dolor in reprehenderit in voluptate velit esse
cillum dolore eu fugiat nulla pariatur. Excepteur sint occaecat cupidatat non
proident, sunt in culpa qui officia deserunt mollit anim id est laborum.
% subsection experi_ncia_de_usu_rio (end)
\subsection{Interface de programação de aplicativos} % (fold)
(idea)Popularmente conhecida como API, o phonegap disponibiliza módulos de
extensão, ou seja, plugins para o acesso a API de várias plataformas nativas e
consequentemente se comunicar com o hardware específico.

Lorem ipsum dolor sit amet, consectetur adipisicing elit, sed do eiusmod
tempor incididunt ut labore et dolore magna aliqua. Ut enim ad minim veniam,
quis nostrud exercitation ullamco laboris nisi ut aliquip ex ea commodo
consequat. Duis aute irure dolor in reprehenderit in voluptate velit esse
cillum dolore eu fugiat nulla pariatur. Excepteur sint occaecat cupidatat non
proident, sunt in culpa qui officia deserunt mollit anim id est laborum.
% subsection interface_de_programa_o_de_aplicativos (end)
% section adobe_phonegap (end)
\section{Appcelerator Titanium} % (fold)
Lorem ipsum dolor sit amet, consectetur adipisicing elit, sed do eiusmod
tempor incididunt ut labore et dolore magna aliqua. Ut enim ad minim veniam,
quis nostrud exercitation ullamco laboris nisi ut aliquip ex ea commodo
consequat. Duis aute irure dolor in reprehenderit in voluptate velit esse
cillum dolore eu fugiat nulla pariatur. Excepteur sint occaecat cupidatat non
proident, sunt in culpa qui officia deserunt mollit anim id est laborum.

Lorem ipsum dolor sit amet, consectetur adipisicing elit, sed do eiusmod
tempor incididunt ut labore et dolore magna aliqua. Ut enim ad minim veniam,
quis nostrud exercitation ullamco laboris nisi ut aliquip ex ea commodo
consequat. Duis aute irure dolor in reprehenderit in voluptate velit esse
cillum dolore eu fugiat nulla pariatur. Excepteur sint occaecat cupidatat non
proident, sunt in culpa qui officia deserunt mollit anim id est laborum.

\subsection{Estrutura Arquitetural} % (fold)
Lorem ipsum dolor sit amet, consectetur adipisicing elit, sed do eiusmod
tempor incididunt ut labore et dolore magna aliqua. Ut enim ad minim veniam,
quis nostrud exercitation ullamco laboris nisi ut aliquip ex ea commodo
consequat. Duis aute irure dolor in reprehenderit in voluptate velit esse
cillum dolore eu fugiat nulla pariatur. Excepteur sint occaecat cupidatat non
proident, sunt in culpa qui officia deserunt mollit anim id est laborum.

Lorem ipsum dolor sit amet, consectetur adipisicing elit, sed do eiusmod
tempor incididunt ut labore et dolore magna aliqua. Ut enim ad minim veniam,
quis nostrud exercitation ullamco laboris nisi ut aliquip ex ea commodo
consequat. Duis aute irure dolor in reprehenderit in voluptate velit esse
cillum dolore eu fugiat nulla pariatur. Excepteur sint occaecat cupidatat non
proident, sunt in culpa qui officia deserunt mollit anim id est laborum.
% subsection estrutura_arquitetural (end)

\subsection{Experiência de usuário} % (fold)
Lorem ipsum dolor sit amet, consectetur adipisicing elit, sed do eiusmod
tempor incididunt ut labore et dolore magna aliqua. Ut enim ad minim veniam,
quis nostrud exercitation ullamco laboris nisi ut aliquip ex ea commodo
consequat. Duis aute irure dolor in reprehenderit in voluptate velit esse
cillum dolore eu fugiat nulla pariatur. Excepteur sint occaecat cupidatat non
proident, sunt in culpa qui officia deserunt mollit anim id est laborum.

Lorem ipsum dolor sit amet, consectetur adipisicing elit, sed do eiusmod
tempor incididunt ut labore et dolore magna aliqua. Ut enim ad minim veniam,
quis nostrud exercitation ullamco laboris nisi ut aliquip ex ea commodo
consequat. Duis aute irure dolor in reprehenderit in voluptate velit esse
cillum dolore eu fugiat nulla pariatur. Excepteur sint occaecat cupidatat non
proident, sunt in culpa qui officia deserunt mollit anim id est laborum.
% subsection experi_ncia_de_usu_rio (end)
\subsection{Interface de programação de aplicativos} % (fold)
(idea)Popularmente conhecida como API, o phonegap disponibiliza módulos de
extensão, ou seja, plugins para o acesso a API de várias plataformas nativas e
consequentemente se comunicar com o hardware específico.

Lorem ipsum dolor sit amet, consectetur adipisicing elit, sed do eiusmod
tempor incididunt ut labore et dolore magna aliqua. Ut enim ad minim veniam,
quis nostrud exercitation ullamco laboris nisi ut aliquip ex ea commodo
consequat. Duis aute irure dolor in reprehenderit in voluptate velit esse
cillum dolore eu fugiat nulla pariatur. Excepteur sint occaecat cupidatat non
proident, sunt in culpa qui officia deserunt mollit anim id est laborum.
% subsection interface_de_programa_o_de_aplicativos (end)
% section appcelerator_titanium (end)
\chapter{Critérios Comparativos} % (fold)
\label{cha:criterios_comparativos}
Lorem ipsum dolor sit amet, consectetur adipisicing elit, sed do eiusmod
tempor incididunt ut labore et dolore magna aliqua. Ut enim ad minim veniam,
quis nostrud exercitation ullamco laboris nisi ut aliquip ex ea commodo
consequat. Duis aute irure dolor in reprehenderit in voluptate velit esse
cillum dolore eu fugiat nulla pariatur. Excepteur sint occaecat cupidatat non
proident, sunt in culpa qui officia deserunt mollit anim id est laborum.

\section{Plataformas Móveis} % (fold)
\label{sec:suporte_a_plataformas_moveis}
Lorem ipsum dolor sit amet, consectetur adipisicing elit, sed do eiusmod
tempor incididunt ut labore et dolore magna aliqua. Ut enim ad minim veniam,
quis nostrud exercitation ullamco laboris nisi ut aliquip ex ea commodo
consequat. Duis aute irure dolor in reprehenderit in voluptate velit esse
cillum dolore eu fugiat nulla pariatur. Excepteur sint occaecat cupidatat non
proident, sunt in culpa qui officia deserunt mollit anim id est laborum.
% section suporte_a_plataformas_móveis (end)

\section{Acesso aos recursos da API nativa} % (fold)
\label{sec:acesso_aos_recursos_da_api_nativa}
Lorem ipsum dolor sit amet, consectetur adipisicing elit, sed do eiusmod
tempor incididunt ut labore et dolore magna aliqua. Ut enim ad minim veniam,
quis nostrud exercitation ullamco laboris nisi ut aliquip ex ea commodo
consequat. Duis aute irure dolor in reprehenderit in voluptate velit esse
cillum dolore eu fugiat nulla pariatur. Excepteur sint occaecat cupidatat non
proident, sunt in culpa qui officia deserunt mollit anim id est laborum.
% section acesso_aos_recursos_da_api_nativa (end)

\section{Performance} % (fold)
\textit{(cite Building Hybrid)}
You may experience potential performance issues because JavaScript is fundamentally
single-threaded, which means that only one operation can be performed at a
time. However, if done right, you can come up with a solution wherein you can
offload background tasks to a native thread, which would execute in parallel while
your app is busy performing UI operations. The native thread would then notify
the JavaScript of the events and task completions/failures.

Lorem ipsum dolor sit amet, consectetur adipisicing elit, sed do eiusmod
tempor incididunt ut labore et dolore magna aliqua. Ut enim ad minim veniam,
quis nostrud exercitation ullamco laboris nisi ut aliquip ex ea commodo
consequat. Duis aute irure dolor in reprehenderit in voluptate velit esse
cillum dolore eu fugiat nulla pariatur. Excepteur sint occaecat cupidatat non
proident, sunt in culpa qui officia deserunt mollit anim id est laborum.
% section performance (end)
\subsection{Uso de Memória} % (fold)
\label{sub:uso_de_memoria}
Lorem ipsum dolor sit amet, consectetur adipisicing elit, sed do eiusmod
tempor incididunt ut labore et dolore magna aliqua. Ut enim ad minim veniam,
quis nostrud exercitation ullamco laboris nisi ut aliquip ex ea commodo
consequat. Duis aute irure dolor in reprehenderit in voluptate velit esse
cillum dolore eu fugiat nulla pariatur. Excepteur sint occaecat cupidatat non
proident, sunt in culpa qui officia deserunt mollit anim id est laborum.
\subsection{CPU} % (fold)
\label{sub:cpu}
Lorem ipsum dolor sit amet, consectetur adipisicing elit, sed do eiusmod
tempor incididunt ut labore et dolore magna aliqua. Ut enim ad minim veniam,
quis nostrud exercitation ullamco laboris nisi ut aliquip ex ea commodo
consequat. Duis aute irure dolor in reprehenderit in voluptate velit esse
cillum dolore eu fugiat nulla pariatur. Excepteur sint occaecat cupidatat non
proident, sunt in culpa qui officia deserunt mollit anim id est laborum.
% subsection cpu (end)
\subsection{Consumo de Bateria} % (fold)
\label{sub:consumo_de_bateria}
Lorem ipsum dolor sit amet, consectetur adipisicing elit, sed do eiusmod
tempor incididunt ut labore et dolore magna aliqua. Ut enim ad minim veniam,
quis nostrud exercitation ullamco laboris nisi ut aliquip ex ea commodo
consequat. Duis aute irure dolor in reprehenderit in voluptate velit esse
cillum dolore eu fugiat nulla pariatur. Excepteur sint occaecat cupidatat non
proident, sunt in culpa qui officia deserunt mollit anim id est laborum.
% subsection consumo_de_bateria (end)

% chapter critérios_comparativos (end)
\chapter{Conclusão} % (fold)
\label{cha:conclusao}
% Cap 5 - Conclusão (resumir o que foi mostrado na monografia, apresentar suas conclusões e mostrar alguns trabalhos futuros)
	Lorem ipsum dolor sit amet, consectetur adipisicing elit, sed do eiusmod
	tempor incididunt ut labore et dolore magna aliqua. Ut enim ad minim veniam,
	quis nostrud exercitation ullamco laboris nisi ut aliquip ex ea commodo
	consequat. Duis aute irure dolor in reprehenderit in voluptate velit esse
	cillum dolore eu fugiat nulla pariatur. Excepteur sint occaecat cupidatat non
	proident, sunt in culpa qui officia deserunt mollit anim id est laborum.
% chapter conclusão (end)
\input{chapters/vanets}

% bibliografia
\bibliography{chapters/references}

% caso haja apêndices, use ``\apendice'' antes de inserir
% as secoes
\apendice
% adicione os apendices, caso existam
\input{chapters/apendice-nakagami}

\end{document}

